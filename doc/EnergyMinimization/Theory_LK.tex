%% LyX 2.3.7 created this file.  For more info, see http://www.lyx.org/.
%% Do not edit unless you really know what you are doing.
\documentclass[english]{article}
\usepackage[T1]{fontenc}
\usepackage[latin9]{inputenc}
\usepackage{geometry}
\geometry{verbose,tmargin=2cm,bmargin=2cm,lmargin=2cm,rmargin=2cm}
\usepackage{verbatim}
\usepackage{esint}
\usepackage{babel}
\begin{document}
\title{Novel Passive Ferrofluid Check Valve}
\author{Lisa Kotowski, Veronica Stuckey, Robert L. Read}
\maketitle
\begin{abstract}
This work focuses on a novel passive ferrofluid check valve. The structure
is free of mechanical moving parts and is based on a ferrofluid plug
in a narrow channel.Check valves, also known as one-way valves, are
devices that allow fluids to move in only one direction primarily
to prevent backflow in the system. The valve returns to stasis (closes)
under the influence of a static magnetic field when the pressure on
the ferrofluid is less than the magnetic forces of the ferrofluid
itself. Here, we present a simple design for a passive, normally closed
ferrofluid check valve consisting of a unique channel-and-chamber
geometry, a bolus of ferrofluid, and a static magnetic field. The
flow is determined only by the force differential between the pressure
of the fluid in the channel and the magnetic force of the ferrofluid
at the intersection of the channel and the chamber. Small pumps and
valves enable flow management in microfluidic systems. A novel passive
ferrofluid check valve is presented. The valve consists of only a
unique channel-and-chamber geometry, ferrofluid, and a stationary
magnetic field. The flow is determined only by the inlet and output
pressure, and the magnetic field is completely static. The prototype
valve and experimental setup are explained and performance of the
valves cracking and collapse pressure reported. This initial design
can be used for microfluid handling and lab-on-a-chip applications.

Additionaly we present a theory of operation based on energy minimization
and compare predicted performance to actual performance.
\end{abstract}

\section{Introduction}

This article is a brief report on an initial but functioning design
of a passive ferrofluid check valve (PFCV) that has no moving parts
except for the ferrofluid bolus itself, which is stationary in normal
operation. By passive, the authors mean a check valve that functions
without changes to the magnetic field affecting the bolus, whether
that field is induced by a permanent magne t or an electromagnet.
That is, the flow is determined purely by the difference between the
inlet port pressure and the outlet port pressure. To our knowledge,
no passive ferrofluid check valve has been previously reported, despite
being an active area of research and despite such a valve having significant
advantages for operation and especially fabrication over valves with
moving parts.

A fundamental component of such devices is the check or one-way valve.
Two check valves on either side of a chamber whose volume can vary
creates a positive displacement pump. A perfect check valve opens
or cracks with minimal pressure on the inlet side and sustains maximal
pressure on the outlet side before collapse, allowing fluid to flow
in only one direction. Following {[}Ref 7{]} we call the maximum pressure
differential the valve can resist in the direction it is intended
to check or block (from outlet to inlet) the sustainable or collapse
pressure.

\section{Theory and Background}

This research posits that a ferrofluid confined to a thin channel
followed by an open area can function effectively as a check valve.
Check valves are passive, one-way valves which prevent the backflow
of liquid. There is a minimum differential upstream pressure, known
as cracking pressure, between the inlet and outlet that will operate
the valve. Check valves operate under the assumption posited below
in 

\begin{equation}
|\vec{F_{m}}|\geq P_{f}*A
\end{equation}
where $|\vec{F_{m}}|$ is the magnitude of the force exerted by the
ferrofluid, $P_{f}$ is the pressure of the fluid against the valve,
and $A$ is the surface area of the ferrofluid plug in contact with
the transport fluid in channel. This leads to cracking pressure, which
is when the pressure of the fluid against the ferrofluid exceeds the
magnetic force of the ferrofluid defined below as $P_{c}$ in 

\begin{equation}
P_{c}>\frac{A}{|\vec{F_{m}}|}
\end{equation}
In the system presented in this research the substance will flow through
the ferrofluid plug when the cracking pressure is exceeded resulting
in a pressure release and preventing the backflow of fluid into the
system. Check valves have applications in microelectronics and medical
devices. The valve presented in this research can be adjusted by changing
the magnetic field, and thus the force exerted by the ferrofluid.
This research focuses on exploring, both theoretically and experimentally,
the simplest case of a static magnetic field.

A cylindrical neodimnium magnet of {*}insert strength and dimensions
here{*} was selected for the purposes of this experiment, and a T-shaped
valve containing a bolus of ferrofluid at the center of a cylindrical
magnetic field. {*}Fig 1 of system mock up{*} %
\begin{comment}
magnetic field of a static cylindrical magnet -> place fluid at one
end of the magnet to form a nearly vertical magnetic field
\end{comment}

Ferrofluids is a colloidal mixture of nanoscale ferromagnetic particles
suspended in a carrier fluid (typically an mineral oil or other organic
solvent). A surfactant coats each particle to prevent the particles
from clumping together, maintaining the homogenous colloidal mixture
as a fluid. The magnetica attraction of tiny nanoparticles is weak
enough to allow the Van der Waals force to prevent magnetic clumping,
and typically do not retain magnetization in the absence of an externally
applied magnetic field\cite{key-1}. Ferrofluid in the presence of
an external magnetic field will produce the force of many dipoles
interacting with one-another and can be summarized using Equation
(3)\cite{key-2}.

\begin{equation}
\vec{F_{ff}=\int_{V}\mu_{0}(\vec{M}\cdot\nabla)\vec{H}dV}+\Sigma_{n}\ointop_{S}\frac{\mu_{0}}{2}(\vec{M_{n}})^{2}dS\oiint
\end{equation}
where $\vec{M}$ is the magnetization of the ferrofluid in an applied
magnetic field, $\vec{H}$is the applied magnetic field, and $\vec{M_{n}}$
is the magnetization upon upon each ferrofluid particle due to one
another. The first volume integral takes into account how the applied
magnetic field will change across the volume of the ferrofluid, and
the second surface integral only looks at the surface of each particle
affecting one another. These can be evaluated in a case-by-case basis
for each applied magnetic field.
\begin{thebibliography}{1}
\bibitem{key-1}Voit, W.; Kim, D. K.; Zapka, W.; Muhammed, M.; Rao,
K. V. (21 March 2011). \textquotedbl Magnetic behavior of coated
superparamagnetic iron oxide nanoparticles in ferrofluids\textquotedbl .
MRS Proceedings. 676. doi:10.1557/PROC-676-Y7.8.

\bibitem{key-2}Rosensweig,�R.�E.�(2013).�Ferrohydrodynamics.�United
States:�Dover Publications.

\end{thebibliography}

\end{document}
